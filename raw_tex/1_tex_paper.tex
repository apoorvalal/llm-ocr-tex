\documentclass[12pt,reqno]{amsart} % add titlepage param for separate title page
\input{mpaper_boilerplate}

\begin{document}

$\psi \in \mathcal{L}_4(P)$ because of $g \in \mathcal{L}_4(P)$. For
the last summand use\[|\psi(x+ct)1_{\{t|\psi| >
\frac{1}{2}\}}|(x+ct) \leq g(x)1_{\{t|g| > \frac{1}{2}\}}(x)\]for
$t$ sufficiently small.

Proposition. Let $T(P,\mathfrak{P})$ be generated by the paths with
remainder terms $r_t$ fulfilling $P(|r_t|) = o(t^0)$.Then every $g \in
T(P,\mathfrak{P})$ is of the form $cl'(\cdot, P) + \psi$ with $\psi
\in \Psi(P)$.{\it Proof.} To simplify our notations, we
assume that $m(P) = 0$ and write $l'=l'(\cdot,P)$.

(i)
Let $g \in T(P,\mathfrak{P})$. There exists a path $P_t$ in
$\mathfrak{P}$ with $\lambda$-density\[P_t = p(1 + t(g+r_t))
\text{ such that } P(|r_t|) = o(t^0), \text{ i.e.,}\]\[P(|t^{-1}
(\frac{p_t}{p} - 1) - g|) = o(t^0).\]Hence there exists a sequence
$t_n \downarrow 0$ such that\[t_n^{-1} (\frac{p_{t_n}}{p} - 1)
\rightarrow g \quad P\text{-a.e.,}\]so
that\begin{equation*}t_n^{-1} (p_{t_n} - p) \rightarrow pg \quad
\lambda \text{-a.e.}\end{equation*}For notational convenience we
write $p_n$ for $p_{t_n}$ and $m_n$ for
$m(P_{t_n})$.

(ii) We start with the remark that $m_n
\rightarrow 0$. Since $t_n^{-1} (p_{t_n} - p) \rightarrow gp$
$\lambda$\text{-a.e., we have} $p_n \rightarrow p$
$\lambda$\text{-a.e. Assume first that} $m_n \rightarrow \infty$ for
some subsequence. Let $a > 0$ be so large that $P(-\infty, a) >
\frac{1}{2}$. Since $m_n > a$ for sufficiently large $n$, symmetry of
$P_n$ about $m_n$ implies $P_n(-\infty, a) \leq P_n(-\infty, m_n) \leq
\frac{1}{2}$. Since $P_n^1(-\infty, a) \rightarrow p^1 (-\infty, a)$
$\lambda$\text{-a.e., Fatou's lemma implies} $P(-\infty, a) \leq
\underline{\lim} P_n(-\infty, a)$, which is a contradiction. Hence
$m_n$, $n \in \mathbb{N}$, is bounded. Let $m_{n'}$, $n \in
\mathbb{N}_0$, be a subsequence converging to $c$, say. Since $p$ is
continuous and $p_n \rightarrow p$ $\lambda$ \text{-a.e.} Lemma 19.1.3
implies the existence of a subsequence such that $p_n(x+m_n)
\rightarrow p(x+c)$ for $\lambda$\text{-a.a. } $x \in \mathbb{R}$.
Since $x \rightarrow p_n(x+m_n)$ is symmetric about $0$ and $x
\rightarrow p(x+c)$ is symmetric.

\pagebreak

$\psi \in \mathcal{L}_4(P)$  because of  $g \in \mathcal{L}_4(P)$. For the last summand use
\[|\psi(x+ct)1_{\{|t|\psi|>\frac{1}{2}\}}| (x+ct) \leq
g(x)1_{\{|t|g|>\frac{1}{2}\}}(x)\]for t sufficiently
small.\textbf{2.3.3. Proposition.} Let $T(P,\mathfrak{B})$ be
generated by the paths with remainder terms $r_t$ fulfilling $P(|r_t|)
= o(t^0)$.Then every $g \in T(P, \mathfrak{B})$ is of the form
$cl'(\cdot, P) + \psi$ with $\psi \in \Psi(P)$.\textit{Proof.} To
simplify our notations, we assume that $m(P) = 0$ and write $l' =
l'(\cdot, P)$.(i) Let $g \in T(\dot{P},\mathfrak{B})$. There exists a
path $P_t$ in $\mathfrak{B}$ with $\lambda$-density\[P_t = p(1 + t(g +
r_t)) \text{ such that } P(|r_t|) = o(t^0), \text{
i.e.,}\]\[P(|t^{-1}(\frac{p_t}{p} - 1) - g|) = o(t^0).\]Hence there
exists a sequence $t_n \downarrow 0$ such
that\[t_n^{-1}(\frac{p_{t_n}}{p} - 1) \to g \quad P\text{-a.e.},\]so
that(2.3.4)\[t_n^{-1}(p_{t_n} - p) \to pg \quad
\lambda\text{-a.e.}\]For notational convenience we write $p_n$ for
$p_{t_n}$ and $m_n$ for $m(P_{t_n})$.(ii) We start with the remark
that $m_n \to 0$. Since $t_n^{-1}(p_n - p) \to gp$$\lambda$-a.e., we
have $p_n \to p$ $\lambda$-a.e. Assume first that $m_n \to \infty$ for
some subsequence. Let $a > 0$ be so large that $P(-\infty, a) >
\frac{1}{2}$. Since $m_n > a$ for sufficiently large $n$, symmetry of
$P_n$ about $m_n$ implies $P_n(-\infty, a) \leq P_n(-\infty, m_n) \leq
\frac{1}{2}$. Since $P_n^1(-\infty, a) \to p^1(-\infty, a)$
$\lambda$-a.e., Fatou's lemma implies $P(-\infty, a) \leq
\underline{\lim} P_n(-\infty, a)$, which is a contradiction. Hence
$m_n$, $n \in \mathbb{N}$, is bounded. Let $m_{n'}$, $n \in
\mathbb{N}_0$, be a subsequence converging to $c$, say. Since $p$ is
continuous and $p_n \to p$ $\lambda$-a.e., Lemma 19.1.3 implies the
existence of a subsequence such that $p_n(x + m_n) \to p(x+c)$ for
$\lambda$-a.a. $x \in \mathbb{R}$. Since $x \to p_n(x + m_n)$ is
symmetric about $0$ and $x \to p(x+c)$ is symmetric

\end{document}
