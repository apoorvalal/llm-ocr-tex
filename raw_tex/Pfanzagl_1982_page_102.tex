 For similar terms of higher order observe that on D the functions $g_i$ are bounded by $N(||g_i||)$, and $r_i$ by $1$.  6.2.16. Remark. Let us now consider the case $Q_2 = P$. Writing $Q$ for $Q_1$ and $g$ for $g_1$, we obtain from Proposition 6.2.2 the existence of a null-function $N$ such that for all p-measures $Q$ fulfilling conditions (6.2.3) - (6.2.5),  (6.2.17) $\mid H(Q,P)^2 - ||g||^2\mid \leq ||g||^2N(||g||)$.  The interpretation: The decomposition of $q/p - 1$ into the sum $g+r$ with $g$ and $r$ fulfilling conditions (6.2.3) - (6.2.5) is to a certain extent arbitrary. But for all such decompositions leading to a function $g$ with $||g||$ small, this amount agrees closely with the Hellinger dis- tance $H(Q,P)$. This explains why the apparently arbitrary component $g$ occurs through $||g||$ in results like Remark 8.4.5 on the slope of the as. envelope power function.  6.2.18. Proposition. Assume that $\mathfrak{W}$ is at $P$ approximable by $T_s(P,\mathfrak{W})$, and that $\overline{\mathfrak{W}}$ is at $P$ approximable by $T_s(P,\overline{\mathfrak{W}})$, i.e. every $Q\in \mathfrak{B}$ admits a $P$-density $1+g+r$ with $g\in T_s(P,\mathfrak{W})$ and $||r|| = o(\Delta(Q;P))$ uniformly for $Q\in \mathfrak{B}$. Assume that $\delta$ approximates $\Delta$ at $P$. Then (6.2.19) $\delta(Q,\overline{\mathfrak{W}})^2 \geq ||g-\overline{g}||^2 + o(\Delta(Q;P)^2)$, where $\overline{g}$ denotes the projection of $g$ into $T_s(P,\overline{\mathfrak{W}})$.  Addendum. If $\overline{\mathfrak{W}}$ fulfills (1.1.11), then equality holds in (6.2.19).  Proof. (i) We have to show that uniformly for $P'\in \overline{\mathfrak{W}}$, (6.2.20) $\delta(Q,P')^2 \geq ||g-\overline{g}||^2 + o(\Delta(Q;P)^2)$.  Since $\mathfrak{W}$ is at $P$ approximable by $T_s(P,\mathfrak{B})$, every $Q \in \mathfrak{B}$ admits a $P$-density $1+g+r$ with $g\in T_s(P,\mathfrak{W})$ and $||r|| = o(\Delta(Q;P))$, uniformly for $Q\in \mathfrak{B}$. Since $\overline{\mathfrak{W}}$ is at $P$ approximable by $T_s(P,\overline{\mathfrak{W}})$, every $P' \in \overline{\mathfrak{W}}$ admits a $P$-den- sity $1+g'+r'$ with $g'\in T(P,\overline{\mathfrak{W}})$ and $||r'|| = o(\Delta(P';P))$. Since $\delta$ 