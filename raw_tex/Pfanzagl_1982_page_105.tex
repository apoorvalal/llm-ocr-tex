 \documentclass{article} \usepackage{amsmath, amssymb} \begin{document}  \begin{equation} \delta(Q^2,P^2) = \sqrt{2} \, \delta(Q,P) \left(1 + o\left(\delta(Q,P)^2\right)\right). \end{equation} (Since these relations hold true for $\delta = \Delta$, any distance function $\delta$ approximating $\Delta$ at $P$ in the sense of Definition 6.2.1 fulfills relations (6.4.3) and (6.4.4) up to terms of order $o(\Delta(Q;P))$.)  Moreover, we mention that one obtains from (6.4.1) and (6.4.2) for sequences of p-measures $Q_n$ with P-density $1 + n^{-1/2} g + n^{-1/2} r_n$ un- der the assumption $P(r_n^2) = o(n^{\circ})$ that \begin{equation} \Delta(Q_n; P_n)^2 = \exp[P(g^2)] - 1 + o(n^{\circ}) \end{equation} and \begin{equation} H(Q_n^n, P_n)^2 = 8(1 - \exp[-\frac{1}{8} P(g^2)]) + o(n^{\circ}). \end{equation}  Though simple relations like (6.4.1) and (6.4.2) do not hold for the variational distance, Reiss (1980, p. 100) proves (for the case $r_n = 0$) a similar asymptotic result, namely \begin{equation} V(Q_n^n, P_n) = 2\Phi\left(\frac{1}{2} P(g^2)^{1/2}\right) - 1 + o\left(n^{-1/2}\right). \end{equation}  We mention these results for the following reason. The as. en- velope power function (see Corollary 8.4.4) depends on the alterna- tive $Q$ through $n^{1/2} \delta(Q,\mathcal{P}_0)$. For the alternatives $Q_n$ defined above, we have $n^{1/2} \delta(Q_n, P) = P(g^2)^{1/2} + o(n^{\circ})$. Hence $n^{1/2} \delta(Q_n, P)$ is of the same order of magnitude as $\delta(Q_n^n, P_n)$. This might suggest to consider the as. envelope power function as a function of $\delta(Q_n^n, P_n)$ (with $\mathcal{P}_0^n = \{P^n: P \in \mathcal{P}_0\}$) rather than $n^{1/2} \delta(Q, \mathcal{P}_0)$. However, with relations (6.4.5) - (6.4.7) being rather complex, this would lead to a formula for the as. envelope power function which looks much more complicated than the one given in Corollary 8.4.4.  \end{document} 