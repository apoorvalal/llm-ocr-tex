 approximates $\Delta$ at P and $||g|| \geq ||g-\bar{g}||$, we obtain uniformly for $Q \in \mathfrak{P}$ \begin{align*} \delta(Q,P)^2 &= ||g||^2 + o(\Delta(Q;P)^2) \\ &\geq ||g-\bar{g}||^2 + o(\Delta(Q;P)^2). \end{align*} Hence (6.2.20) holds for $P' \in \overline{\mathfrak{P}}$ with $\delta(Q,P') \geq \delta(Q,P)$. If $\delta(Q,P')$ $\leq \delta(Q,P)$, then $\delta(P',P) \leq 2\delta(Q,P)$. Since $\delta$ approximates $\Delta$ at $P$, we obtain uniformly for $Q \in \mathfrak{P}$ \begin{align*} \delta(Q,P') &= \Delta(Q,P';P)^2 + o(\Delta(Q;P)^2) \\ &= ||g-g'||^2 + o(\Delta(Q;P)^2) \\ &\leq ||g-\bar{g}||^2 + o(\Delta(Q;P)^2). \end{align*} (ii) Since $\bar{g} \in T_s(P,\overline{\mathfrak{P}})$, condition (1.1.10) implies the existence of $p^* \in \overline{\mathfrak{P}}$ with P-density $1+\bar{g}+r^*$ such that $||r^*|| = o(||g||) = o(\Delta(Q;P))$. This holds uniformly for $\bar{g} \in T(P,\overline{\mathfrak{P}})$ and, therefore, uniformly for $Q \in \mathfrak{P}$. Since $\delta$ approximates $\Delta$ at P and $\Delta(P^*;P) = o(\Delta(Q;P))$, we obtain uni- formly for $Q \in \mathfrak{P}$ \begin{align*} \delta(Q,P^*)^2 &= \Delta(Q,P^*;P)^2 + o(\Delta(Q;P)^2) + o(\Delta(P^*;P)^2) \\ &= ||g-\bar{g}||^2 + o(\Delta(Q;P)^2). \end{align*}  6.3. Distances in parametric families  For parametric families, an expression for the distance between p-measures in terms of the parameters will be useful.  6.3.1. Proposition. Let $\mathfrak{P} = \{P_\theta: \theta \in \Theta\}, \Theta \subset \mathbb{R}^k$. Assume that the den- sities admit continuous partial derivatives such that for $i = 1,...,k$ and for $|\tau-\theta| < \varepsilon$, say, $$|p^{(i)}(x,\tau) - p^{(i)}(x,\theta)| \leq |\tau-\theta|p(x,\theta)M(x,\theta),$$ where $M(\cdot,\theta)$ and $\ell^{(i)}(\cdot,\theta)$ are $P_\theta$-square integrable and $L(\theta) := P_\theta(\ell^{(\cdot)}(\cdot,\theta)\ell^{(\cdot)}(\cdot,\theta)')$ is nonsingular. Then the following relatior 