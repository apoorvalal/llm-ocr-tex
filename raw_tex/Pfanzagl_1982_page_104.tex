 holds uniformly for τ',τ" ΕΘ: Δ(Ρτι,Ρτ:Ρ) = 1/2 ((τ'-") 'L(θ) (τ' -τ")) + τ' -τ" |Ο (τ'−8 | + |τ"-8|).  Proof. A Taylor expansion about t and hence = 1 (τ'+τ") /2 yields p(x,t) = p(x,t)+(t'-") jp (*) (x, (1+u)t'/2 + (1-u)"/2)du 1 P(x,t) = p(x,t) + ("-") }p(*) (x, (1+) "/2 + (1-u)r'/2)du, (τ"-τ')'p 1 p(x,t')-p(x,τ") = (τ'-τ")'jp(*) P (x, (1+u) τ' /2 + (1-u) τ"/2) du = )τ'-τ"(יp)*( (x,θ) + (τ'-τ") 'r(x, θ,τ', τ") 1 (i) with r₁ (x,θ,τ',τ"):= {(p (i) (x, (1+u) τ'/2 + (1-и)т"/2) - p (x,0)) du. Since r₁ (x,θ,τ',τ") | ≤ (τ'-θ| + |τ"-8|)p(x,0) M(x,0), the asser- i tion follows easily.  6.4. Distances for product measures  Since our asymptotic results refer to product measures, it appears natural to consider how the distances between product measures depend on the distances between their components. Using relations (6.1.4) and (6.1.2) we obtain (6.4.1) and (6.4.2) m m Π (1 + Δ (21:21) 2) m 1+ 1 + ( X i=1 i 2 X P.) i i = 1 = m i=1 m P 1-(2)²= (1 - (21P)2). i=1 i m X i=1 Π i=1 H(2₁,P₁)). i From these relations we immediately obtain that both distances, A and H, fulfill the relations (6.4.3) and δ (QXM,PXM) = δ (Q,P) 