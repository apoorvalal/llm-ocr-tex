 and by applying the following relations to the remainder terms. (ii) We have $A_j^C \cup B_j^C \subseteq A_i^C \cup A_i A_j^C \cup A_i B_j^C$. Hence by (6.2.7), (6.2.6) and (6.2.9), \begin{equation*}  P(|g_i| \mathbb{1}_{A_j^C \cup B_j^C}) \leq ||g_1||^2 N(||g_1||) + ||g_2||^2 N(||g_2||). \end{equation*} By (6.2.8), (6.2.6) and (6.2.9), \begin{equation*}  P(g_i^2 \mathbb{1}_{A_j^C \cup B_j^C}) \leq ||g_1||^2 N(||g_1||) + ||g_2||^2 N(||g_2||). \end{equation*} Using $A_j^C \cup B_j^C \subseteq B_i^C \cup B_i A_j^C \cup B_i B_j^C$ we obtain from (6.2.10), (6.2.6) and (6.2.9) \begin{equation*}  P(|r_i| \mathbb{1}_{A_j^C \cup B_j^C}) \leq ||g_1||^2 N(||g_1||) + ||g_2||^2 N(||g_2||). \end{equation*} (iii) From (6.2.13) and (6.2.15), together with (6.2.6) and (6.2.9), we obtain \begin{equation*} P(((1+g_1 +r_1)(1+g_2 +r_2))^{1/2} \mathbb{1}_{D^C}) \end{equation*} \begin{equation*} \leq (P((1+g_1 +r_1) \mathbb{1}_{D^C}))^{1/2} (P((1 + g_2 + r_2) \mathbb{1}_{D^C}))^{1/2} \end{equation*} \begin{equation*} \leq ||g_1||^2 N(||g_1||) + ||g_2||^2 N(||g_2||). \end{equation*} From (6.2.14) we obtain \begin{equation*} P(((g_1 - g_2)^2 \mathbb{1}_{D^C}) \leq ||g_1||^2 N(||g_1||) + ||g_2||^2 N(||g_2||). \end{equation*} (iv) It remains to show that we can neglect the $P$-integrals over $D$ of the remaining terms in (6.2.12). From (6.2.13) and (6.2.15), together with $P(g_i) = P(r_i) = 0$, we obtain \begin{equation*}  P(g_i \mathbb{1}_D) \leq ||g_1||^2 N(||g_1||) + ||g_2||^2 N(||g_2||), \end{equation*} \begin{equation*}  P(r_i \mathbb{1}_D) \leq ||g_1||^2 N(||g_1||) + ||g_2||^2 N(||g_2||). \end{equation*} Furthermore, we apply (6.2.11) to \begin{equation*}  P(|g_i r_j| \mathbb{1}_D) \leq ||g_i|| ||r_j \mathbb{1}_{B_j}||, \end{equation*} \begin{equation*}  P(|r_i r_j| \mathbb{1}_D) \leq ||r_i \mathbb{1}_{B_i}|| ||r_j \mathbb{1}_{B_j}||. \end{equation*} 